\chapter{Conclusion}
\label{chap:conclusion}
This thesis began with a central question: \textit{Is it possible to develop a system that facilitates academic records sharing between institutions, while also enabling students to truly own and manage their data?} Over the course of this work, we presented our proposed solution to this challenge, highlighting its strengths and acknowledging its limitations.

We began by reviewing related work in the field, identifying both useful foundations and critical shortcomings. These insights guided the definition of a comprehensive use case, capturing the multifaceted nature of the problem. From this use case, we derived a set of functional and non-functional requirements that our system needed to fulfil.
We then explored the design of the on-chain components, providing an in-depth discussion of the smart contracts and the rationale behind key architectural decisions. Particular attention was given to the adoption of account abstraction, a pivotal choice that enabled us to simplify user interaction with the blockchain while maintaining security and decentralization.
Following this, we detailed the design of the off-chain components—including the \gls{sdk}, browser extension, decentralized storage system, and \gls{cli}—explaining how these elements interface with the smart contracts. We also illustrated how account abstraction is implemented across the system, and how decentralized storage helps reduce blockchain-related costs while preserving the system’s decentralized nature.
The implementation chapter offered a deeper technical look at the system's core, helping readers understand how the various modules operate in practice. We validated our solution through testing, confirming its functionality and cost-efficiency in a controlled environment. Finally, we discussed the broader implications of our work and identified promising directions for future development.

In summary, this thesis contributes a practical and innovative solution to the management of academic records, enabling universities to align with Web3 principles such as decentralization and data ownership. Beyond its application in the educational sector, our work also offers broader value to the blockchain community by demonstrating how to integrate off-chain and on-chain components in a cohesive, user-friendly system