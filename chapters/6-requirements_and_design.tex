\chapter{Requirements and Design}

\section{Requirements}
In this section, our aim is to describe both the functional and non-functional requirements that guided us in the development of the comprehensive academic wallet system. These requirements were derived from an in-depth analysis of the use cases, as well as the specific needs and demands of the various stakeholders engaged with the system. This stakeholder group includes the administrator of the \acrshort{ew} system, students, and universities.

\subsection{Functional Requirements}
The primary functional requirement for the system administrator is the ability to register universities following their subscription request. This process must be preceded by a thorough verification of the provided data, in order to ensure that only trusted institutions are granted access to the system.

Universities must be able to initiate a subscription by submitting their institutional information. Upon approval, they require secure authentication mechanisms to access the system and exercise their privileges. Authentication is a critical precondition for nearly all operations, including the viewing and modification of students' academic records. These operations may involve retrieving the list of courses attended by a student or issuing a new certification. Moreover, authenticated universities are responsible for creating \acrshort{sw} for newly enrolled students who do not yet posses one, as well as requesting permission from students to access their academic records. Delegating the creation of student \acrshort{sw} to universities accelerates the data verification process, as the system relies on the institution's trustworthiness to validate student information. Without proper authentication, access to both student and university data, whether for viewing or modification, must be strictly restricted. To support seamless integration with university \acrshort{lms}, the system should also expose comprehensive \acrshort{api} endpoints, enabling universities to interact with and utilize its full range of features.

Students' functional requirements primarily revolve around accessing and managing their academic records. The \acrshort{ew} system must enable students to securely authenticate and access their \acrshort{sw} via a web interface, which also provides the ability to view and manage their data. Once authenticated, students should be able to grant universities explicit permissions to access their records, with fine-grained control over viewing and modification rights. This capability is particularly crucial in enabling institutions that are not the student's home university to view and act upon their academic data when necessary.

\subsection{Non-Functional Requirements}
