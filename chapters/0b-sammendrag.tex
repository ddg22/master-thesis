\chapter*{Sammendrag}
Mens verden raskt går fra tradisjonelle Web 2.0-systemer, dominert av store selskaper som tjener penger på sentraliserte data, til Web3-løsninger basert på blockchain-teknologiens desentraliserte natur, er akademia fortsatt i stor grad avhengig av sentraliserte databaser og statiske dokumenter, enten digitale eller på papir. Denne utdaterte tilnærmingen hindrer deling av akademiske opplysninger mellom institusjoner, som ofte krever produksjon, validering og verifisering av dokumenter, samt enighet om felles formater.

Dette arbeidet foreslår en løsning for å modernisere forvaltningen av akademiske opptegnelser. Ved å utnytte blockchain og Ethereum smart contracts legger vi grunnlaget for et system basert på sikre og manipulasjonssikre teknologier. Gjennom bruk av kontoabstraksjon er vi i stand til å utvikle en brukervennlig løsning som muliggjør gasless interaksjoner for brukerne. EduWallet er avhengig av et Software Development Kit (SDK) som gjør det mulig for universiteter å samhandle med on-chain-funksjonaliteter, og en nettleserutvidelse som gjør det mulig for studenter å administrere sine data. En viktig funksjon i den foreslåtte løsningen er full dataeierskap, som i sin helhet gis til studentene, som kontrollerer tilgangen til sine akademiske opplysninger. I tillegg brukes et desentralisert lagringssystem til å lagre studentenes sertifikater, noe som minimerer behovet for lagring av data på kjeden.

EduWallet har som mål å tilby et omfattende miljø innenfor akademiske institusjoner, slik at universitetene kan effektivisere byråkratiske prosesser og tilpasse seg de nyeste Web3-standardene. Den foreslåtte løsningen fungerer også som et praktisk eksempel på hvordan man effektivt kan integrere on-chain- og off-chain-komponenter i et sammenhengende system.