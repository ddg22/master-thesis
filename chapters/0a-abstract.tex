\chapter*{Abstract}
While the world is rapidly transitioning from traditional Web 2.0 systems, dominated by large companies monetizing centralized data, toward Web3 solutions built on blockchain’s decentralized nature,  the academic sector remains largely reliant on centralized databases and static documents, whether digital or on paper. This outdated approach hampers the sharing of academic records between institutions, which often requires the production, validation, and verification of documents, as well as agreement on common formats.

This work proposes a solution to modernize academic records management. By leveraging blockchain and Ethereum smart contracts, we lay the foundation for a system based on secure and tamper-proof technologies. Through the adoption of account abstraction, we are able to develop a user-friendly solution that enables gasless interactions for users. EduWallet relies on a Software Development Kit (SDK) that allows universities to interact with on-chain functionalities, and a browser extension that enables students to manage their data. A key feature of the proposed solution is full data ownership, which is entirely granted to students, who control access to their academic records. Additionally, a decentralized storage system is used to store students’ certificates, minimizing the need for on-chain data storage.

EduWallet aims to provide a comprehensive environment within academic institutions, allowing universities to streamline bureaucratic processes and align with the latest Web3 standards. The proposed solution also serves as a practical example of how to effectively integrate on-chain and off-chain components into a cohesive system.